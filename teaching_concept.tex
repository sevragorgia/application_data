%Sergio Vargas R. 2015.

\documentclass[a4paper,11pt]{article}
%\documentclass[a4paper,11pt,twocolumn]{article}%twocolumn layout; not sure if this works correctly. Feel free to experiment.
\usepackage[pdftex]{color,graphicx}
\usepackage[T1]{fontenc}
\usepackage{pxfonts}

%color links to figs, etc.
%\usepackage[pdftex,colorlinks,urlcolor=blue,breaklinks]{hyperref}
\usepackage[pdftex,colorlinks]{hyperref}
\hypersetup{colorlinks,%
		    citecolor=black,%
		    filecolor=black,%
		    linkcolor=black,%
		    urlcolor=black,%
		    breaklinks=true,%
		    pdftex}
		    
%easily manipulate margins
\usepackage{geometry}
\makeatletter
\if@twocolumn%
	\geometry{twoside,
		paperwidth=210mm,
  		paperheight=297mm,
  		textheight=682pt,
  		textwidth=516pt,
  		centering,
  		headheight=50pt,
  		headsep=12pt,
  		footskip=18pt,
  		footnotesep=24pt plus 2pt minus 12pt,
 		columnsep=14pt}	
\else%if using twocolumns, I still need to modify the textwidth to accomodate the watermark
	\geometry{twoside,
		  paperwidth=210mm,
		  paperheight=297mm,
		  textheight=558pt,
		  textwidth=380pt,
		  centering,
		  headheight=50pt,
		  headsep=12pt,
		  footskip=18pt,
		  footnotesep=24pt plus 2pt minus 12pt}
\fi%

%change the font
\usepackage{charter}

%CC logos to add in the footer
\usepackage{ccicons}

%easily manipulate color
\usepackage{color}

%nicely print bioRxiv
\newcommand{\biorxiv}{\emph{bio{\color{red}R}$\chi$iv}}

%add biorxiv type of article using the package background
\usepackage{tikz}
\usepackage[firstpage=True]{background}

%use either of the following commands \newresults, \confirmatoryresults or \contradictoryresults to define the type of article in the first page
\newcommand{\newresults}{
\makeatletter
\if@twocolumn%
	\backgroundsetup{
		position={0.572\paperwidth,-0.025\paperheight} ,
		angle=270,
		color=black,
		opacity=0.60,
		scale=1.5,
		contents={\tikz\node[text=black,fill=gray!40,align=left, minimum width=2.3cm,minimum height=0.6cm,inner sep=0]{New Results};}}
\else
	\backgroundsetup{
		position={0.338\paperwidth,-0.06\paperheight} ,
		angle=270,
		color=black,
		opacity=0.60,
		scale=2.5,
		contents={\tikz\node[text=black,fill=gray!40,align=left, minimum width=6.0cm,minimum height=0.6cm,inner sep=0]{Salzburg, Evolutionary Zoology};}}
\fi%
}

\newcommand{\confirmatoryresults}{
\makeatletter
\if@twocolumn%
\backgroundsetup{
	position={0.572\paperwidth,-0.03\paperheight} ,
	angle=270,
	color=black,
	opacity=0.60,
	scale=1.5,
	contents={\tikz\node[text=black,fill=gray!40,align=left, minimum width=3.9cm,minimum height=0.6cm,inner sep=0]{Confirmatory Results};}}
\else
	\backgroundsetup{
	position={0.338\paperwidth,-0.07\paperheight} ,
	angle=270,
	color=black,
	opacity=0.60,
	scale=2.5,
	contents={\tikz\node[text=black,fill=gray!40,align=left, minimum width=3.9cm,minimum height=0.6cm,inner sep=0]{Confirmatory Results};}}
\fi%
}

\newcommand{\contradictoryresults}{
\makeatletter
\if@twocolumn%
	\backgroundsetup{
	position={0.572\paperwidth,-0.03\paperheight} ,
	angle=270,
	color=black,
	opacity=0.60,
	scale=1.5,
	contents={\tikz\node[text=black,fill=gray!40,align=left, minimum width=4.0cm,minimum height=0.6cm,inner sep=0]{Contradictory Results};}}
\else
	\backgroundsetup{
	position={0.338\paperwidth,-0.07\paperheight} ,
	angle=270,
	color=black,
	opacity=0.60,
	scale=2.5,
	contents={\tikz\node[text=black,fill=gray!40,align=left, minimum width=4.0cm,minimum height=0.6cm,inner sep=0]{Contradictory Results};}}
\fi%
}

%enhanced floats
\usepackage{float}

%rotate floats if necessary
\usepackage{rotating}

%nicer, clever tables, uncomment if necessary
\usepackage{supertabular}

%tables with notes, etc., uncomment if necessary
%\usepackage{threeparttable}

%improved captions, uncomment if necessary
\usepackage{caption}

%add an author block
\usepackage{authblk}
%redefine authorblock font size and affiliation font size
\renewcommand\Authfont{\large}
\renewcommand\Affilfont{\scriptsize}

%add nice headers
\usepackage{fancyhdr}
\pagestyle{fancy}
\renewcommand{\headrulewidth}{0pt}
\lhead{{\footnotesize Vargas, 2017}}
\chead{}
\rhead{{\footnotesize Salzburg Evolutionary Biology, Teaching Concept}}
\lfoot{\ccbynd}
\cfoot{\thepage}
\rfoot{}

%redefine plain header
\fancypagestyle{plain}{
\renewcommand{\headrulewidth}{0pt}
\lhead{}
\chead{}
\rhead{{\footnotesize}}
\lfoot{}
\cfoot{\ccbynd}
\rfoot{}
}

%add water mark to the bottom right of all pages to make clear this is a preprint
\usepackage{draftwatermark}
\makeatletter
\if@twocolumn
	\SetWatermarkText{Preprint}
	\SetWatermarkScale{0.20}
	\SetWatermarkAngle{270}
	%note: I defined this commands in the draftwatermark.sty file. For some reason they are in teh manual but not in the sty file...
	\SetWatermarkHorCenter{0.97\paperwidth}
	\SetWatermarkVerCenter{-.88\paperheight}
\else
	\SetWatermarkText{Teaching Concept}
	\SetWatermarkScale{0.25}
	\SetWatermarkAngle{270}
	%note: I defined this commands in the draftwatermark.sty file. For some reason they are in teh manual but not in the sty file...
	\SetWatermarkHorCenter{0.95\paperwidth}
	\SetWatermarkVerCenter{-.82\paperheight}
\fi

%uncomment the type of bioRxiv preprint below to added to the first page
\newresults{}
%\confirmatoryresults{}
%\contradictoryresults{}

%flushleft the title, authorblock and Abstract
%modified from http://tex.stackexchange.com/questions/85343/left-align-abstract-title-and-authors
\makeatletter
\renewcommand{\maketitle}{\bgroup\setlength{\parindent}{0pt}
\begin{flushleft}
  \thispagestyle{plain}
  \textbf{\@title}

  \@author
\end{flushleft}\egroup
}
\makeatother

%redefine the abstract
\renewenvironment{abstract}
 {\small
  \begin{flushleft}
  \textbf{\abstractname}\vspace{-0.40em}\vspace{0pt}
  \end{flushleft}
  \list{}{
    \setlength{\leftmargin}{0cm}%
    \setlength{\rightmargin}{\leftmargin}%
  }%
  \item\relax}
 {\endlist}

\hyphenation{}

\renewcommand*{\thefootnote}{\fnsymbol{footnote}}

%if to do notes need to be added: need to configure this!
\usepackage[colorinlistoftodos]{todonotes}

%feel free to change the reference style to suit your needs
\usepackage[firstinits=true, backref=false, maxcitenames=99, sorting=none, hyperref=auto, style=numeric-comp, defernumbers=true, backend=bibtex]{biblatex}[2010/11-19]

%change the name of this file to point to your bib file.
\bibliography{./Bibliography/Literature}

%send footnotes to the endlis
\usepackage{endnotes}

%global no indent
\setlength{\parindent}{0pt}

\begin{document}

%always keep the \newline command at the end of the title to add space between the title and the authors
\title{\Large Teaching Concept\newline}

\author[1]{Sergio Vargas R.\footnote[2]{\href{sergio.vargas@lmu.de}{sergio.vargas@lmu.de}}}
%\author[1,3,4]{Author with multiple affiliations}

\affil[1]{Department of Earth- \& Environmental Sciences, Palaeontology and Geobiology, Ludwig-Maximilians-Universtit\"at M\"unchen, Richard-Wagner Str. 10, D-80333 M\"unchen, Germany}
%\affil[2]{Forschungsinstitut und Naturmuseum Senckenberg, Senckenberganlage 25, D-60325 Frankfurt am Main, Germany}
%\affil[3]{Bavarian State Collections of Palaeontology and Geology, Richard-Wagner Str. 10, D-80333 M\"unchen, Germany}
%\affil[4]{GeoBio-CenterLMU, Richard-Wagner Str. 10, D-80333 M\"unchen, Germany}

\date{}

\maketitle
%\tableofcontents
%\begin{abstract}
%Blah, blah, blah.
%\\

%\textbf{Key words} Blah, Blah blah.
%\end{abstract}

\section*{Teaching experience}

My duties as a ``Wissenschaftlicher Assistent'' at the LMU M\"unchen involve teaching a weekly load of five hours each semester. Below, I will provide a brief account on the courses I have taught and later I will describe the way each course proceeds and what it usually covers.\\

I have been teaching different courses since I started my current position in 2013. I also had the opportunity to participate in the transition to new Bachelor and Master level \emph{curricula}. This way I was able to participate in the conception and design of new courses and get first-hand experience on the process of transforming \emph{curricula} and the associated challenges this involves (e.g.~Bologna-conformity, Module-conception).\\

During the first two semesters I had to teach courses belonging to the previous Master and Bachelor programs. These included a paleozoology laboratory for third semester students of the Bachelor in Geosciences program and an advanced first semester Master-level seminar on Life-Earth Interactions during the Precambrian.\\

Starting on the Winter Semester 2014-2015, I took over all our courses dealing specifically with Data Analysis and developed a laboratory course on Geomicrobiology. For these courses I took the responsibility of developing the lectures/practicals for our new Bachelor and Master programs. These are now organized as a Bachelor level introductory course on data management and data analysis and visualization using R and a Master level lecture on statistics; this lecture is currently being redesigned to include a practical starting on the summer semester of 2018. The Geomicrobiology practical involves a ca.~one week full-time molecular laboratory practical followed by a one or two day data analysis and bioinformatics hands-on tutorial. This course is now taught by Prof.~Dr.~William Orsi, but the concept I proposed (see below) has remained essentially unaltered during the transition.\\

In addition to the above mentioned courses, I am jointly responsible for our Bachelor and Master level course on Evolution, which serve as an introduction to evolutionary biology for students with a core background in Geosciences. Finally, my teaching load also involves contributing on a number of other miscellaneous Master level seminars taught by the faculty and designed to communicate our research to the new students, to teach communication skills to our students, and to increase their research skills.

\section*{Student supervision experience}

A small part of the teaching load goes to supervising bachelor and master students; formally doctoral fellows are not students and their supervision cannot be subtracted from the teaching load. In this regard, since 2013 I have supervised Bachelor and Master level theses, as well as students from the \textbf{Erasmus Mundus Master Programme in Evolutionary Biology}\endnote{The summer of 2013 I supervised two MEME students whose research project resulted in the following publications:\newline\newline\underline{Sergio Tusso}, Kerstin Morcinek, Catherine Vogler, Peter J. Schupp,  Ciemon F. Caballes, \textbf{Sergio Vargas} \& Gert W\"orheide. Genetic structure of the Crown-of-Thorns Starfish Seastar in the Pacific Ocean, with focus on Guam Island. \textbf{\emph{PeerJ}}, 4: e1970 \newline\underline{Sandra L. Ament-Vel\'asquez}, Odalisca Breedy, Jorge Cort\'es, Hector M Guzman, Gert W\"orheide \& \textbf{Sergio Vargas*}. 2016. Homoplasious colony morphology and mito-nuclear phylogenetic discordance among Eastern Pacific octocorals. \textbf{\emph{Molecular Phylogenetics and Evolution}}, 98: 373-381\newline\newline More information on the MEME program can be found at \href{http://www.evobio.eu/cohort-2012}{http://www.evobio.eu/cohort-2012}\newline} and the \textbf{Munich Graduate School for Evolution, Ecology and Systematics}\endnote{In 2015 I supervised three students from the EES program:\newline\newline Raul A. Gonz\'alez Pech: Manuscript in preparation. Gonz\'ales et al. Individual variation in the response of a resilient coral to low pH water exposure. \newline Christian Feregrino: mitogenomics in the family Gorgoniidae, a manuscript including his work is ready for submission to \emph{Molecular Phylogenetics and Evolution}: Angelo Poliseno, \underline{Christian Feregrino}, St\'ephane Sartoretto, Didier Aurelle, Gert W\"orheide, Catherine S. McFadden, \textbf{Sergio Vargas}. Comparative mitogenomics, phylogeny and evolutionary history of \emph{Leptogorgia} (Gorgoniidae).\newline Anna Farr\'e Orteu: EES Student Award for her poster about changes in transcriptomic activity in a sponge upon symbiont loss. \newline\newline More information on the EES Program can be found at \href{http://www.ees.bio.lmu.de/students/ees-students/index.html}{http://www.ees.bio.lmu.de/students/ees-students/index.html}\newline} who have done their \emph{Research Training} in our group under my supervision. Additionally, I have co-supervised two doctoral theses since 2013.\\

Bachelor theses are short research projects taking place during the summer. Master projects also take place during the summer semester but encompass a broader project we typically aim to publish. This is possible thanks to the existence in the program of a 12 ECTs course named \emph{Research Project}, in which master students are supposed to work on a preliminary investigation that should serve as a foundation for their master thesis project.\\

Since most student supervision activities take place during the summer semester ---the exception being the \emph{Research Project}, which takes place during winter---, I have tried to use a cascade model in which the bachelor project is complementary to a master project that is usually part of a broader research project, for instance, a doctoral thesis. This way of working with students has been fruitful so far as judged by the relative success publishing or preparing manuscripts in which our former undergraduate students are either first authors or coauthors\endnote{See Notes 1 and 2}.

\section*{Developed teaching concept and continuing education}

The challenges I faced designing different types of courses at both Bachelor and Master level, and supervising bachelor, master and doctoral students led me to enroll myself in the \underline{PROFI} in der \underline{L}ehre (PROFIL) continuing education program of the LMU. The PROFIL program offer a series of courses and seminars to university staff with a teaching duty. The courses deal with different topics related to the conception of different types of courses, the deployment of different evaluation forms, student counseling, advising, mentoring and coaching, and other soft skills like flipchart creation.

\section*{Similar courses in the Salzburger curriculum}


\section*{Challenges to be faced in Salzburg}


%\todo[inline]{this is how you can add todo notes to your manuscript!...}

%if you want to add the references manually uncomment the following lines
%\section*{References}
%{\noindent
%Your reference here... if you want to write them.
%}

%you also need to comment the line \bibliography{./Bibliography/Literature} in the preamble...

%if you use bibtex use it to print the bibliography
%\printbibliography

%Paste tables and figures here if you want them to appear at the end of the preprint. Otherwise place them wherever you want!

\newpage

%print note at the end
\begingroup
\parindent 0pt
\def\enotesize{\footnotesize}
\theendnotes
\endgroup

%\listoftodos[To Do:]
\end{document}
