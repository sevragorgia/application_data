%Sergio Vargas R. 2015.

\documentclass[a4paper,11pt]{article}
%\documentclass[a4paper,11pt,twocolumn]{article}%twocolumn layout; not sure if this works correctly. Feel free to experiment.
\usepackage[pdftex]{color,graphicx}
\usepackage[T1]{fontenc}
\usepackage{pxfonts}

%color links to figs, etc.
%\usepackage[pdftex,colorlinks,urlcolor=blue,breaklinks]{hyperref}
\usepackage[pdftex,colorlinks]{hyperref}
\hypersetup{colorlinks,%
		    citecolor=black,%
		    filecolor=black,%
		    linkcolor=black,%
		    urlcolor=black,%
		    breaklinks=true,%
		    pdftex}
		    
%easily manipulate margins
\usepackage{geometry}
\makeatletter
\if@twocolumn%
	\geometry{twoside,
		paperwidth=210mm,
  		paperheight=297mm,
  		textheight=682pt,
  		textwidth=516pt,
  		centering,
  		headheight=50pt,
  		headsep=12pt,
  		footskip=18pt,
  		footnotesep=24pt plus 2pt minus 12pt,
 		columnsep=14pt}	
\else%if using twocolumns, I still need to modify the textwidth to accomodate the watermark
	\geometry{twoside,
		  paperwidth=210mm,
		  paperheight=297mm,
		  textheight=558pt,
		  textwidth=380pt,
		  centering,
		  headheight=50pt,
		  headsep=12pt,
		  footskip=18pt,
		  footnotesep=24pt plus 2pt minus 12pt}
\fi%

%change the font
\usepackage{charter}

%CC logos to add in the footer
\usepackage{ccicons}

%easily manipulate color
\usepackage{color}

%nicely print bioRxiv
\newcommand{\biorxiv}{\emph{bio{\color{red}R}$\chi$iv}}

%add biorxiv type of article using the package background
\usepackage{tikz}
\usepackage[firstpage=True]{background}

%use either of the following commands \newresults, \confirmatoryresults or \contradictoryresults to define the type of article in the first page
\newcommand{\newresults}{
\makeatletter
\if@twocolumn%
	\backgroundsetup{
		position={0.572\paperwidth,-0.025\paperheight} ,
		angle=270,
		color=black,
		opacity=0.60,
		scale=1.5,
		contents={\tikz\node[text=black,fill=gray!40,align=left, minimum width=2.3cm,minimum height=0.6cm,inner sep=0]{New Results};}}
\else
	\backgroundsetup{
		position={0.338\paperwidth,-0.06\paperheight} ,
		angle=270,
		color=black,
		opacity=0.60,
		scale=2.5,
		contents={\tikz\node[text=black,fill=gray!40,align=left, minimum width=6.0cm,minimum height=0.6cm,inner sep=0]{Salzburg, Evolutionary Zoology};}}
\fi%
}

\newcommand{\confirmatoryresults}{
\makeatletter
\if@twocolumn%
\backgroundsetup{
	position={0.572\paperwidth,-0.03\paperheight} ,
	angle=270,
	color=black,
	opacity=0.60,
	scale=1.5,
	contents={\tikz\node[text=black,fill=gray!40,align=left, minimum width=3.9cm,minimum height=0.6cm,inner sep=0]{Confirmatory Results};}}
\else
	\backgroundsetup{
	position={0.338\paperwidth,-0.07\paperheight} ,
	angle=270,
	color=black,
	opacity=0.60,
	scale=2.5,
	contents={\tikz\node[text=black,fill=gray!40,align=left, minimum width=3.9cm,minimum height=0.6cm,inner sep=0]{Confirmatory Results};}}
\fi%
}

\newcommand{\contradictoryresults}{
\makeatletter
\if@twocolumn%
	\backgroundsetup{
	position={0.572\paperwidth,-0.03\paperheight} ,
	angle=270,
	color=black,
	opacity=0.60,
	scale=1.5,
	contents={\tikz\node[text=black,fill=gray!40,align=left, minimum width=4.0cm,minimum height=0.6cm,inner sep=0]{Contradictory Results};}}
\else
	\backgroundsetup{
	position={0.338\paperwidth,-0.07\paperheight} ,
	angle=270,
	color=black,
	opacity=0.60,
	scale=2.5,
	contents={\tikz\node[text=black,fill=gray!40,align=left, minimum width=4.0cm,minimum height=0.6cm,inner sep=0]{Contradictory Results};}}
\fi%
}

%enhanced floats
\usepackage{float}

%rotate floats if necessary
\usepackage{rotating}

%nicer, clever tables, uncomment if necessary
\usepackage{supertabular}

%tables with notes, etc., uncomment if necessary
%\usepackage{threeparttable}

%improved captions, uncomment if necessary
\usepackage{caption}

%add an author block
\usepackage{authblk}
%redefine authorblock font size and affiliation font size
\renewcommand\Authfont{\large}
\renewcommand\Affilfont{\scriptsize}

%add nice headers
\usepackage{fancyhdr}
\pagestyle{fancy}
\renewcommand{\headrulewidth}{0pt}
\lhead{{\footnotesize Vargas, 2017}}
\chead{}
\rhead{{\footnotesize Salzburg Evolutionary Biology, Teaching Concept}}
\lfoot{\ccbynd}
\cfoot{\thepage}
\rfoot{}

%redefine plain header
\fancypagestyle{plain}{
\renewcommand{\headrulewidth}{0pt}
\lhead{}
\chead{}
\rhead{{\footnotesize}}
\lfoot{}
\cfoot{\ccbynd}
\rfoot{}
}

%add water mark to the bottom right of all pages to make clear this is a preprint
\usepackage{draftwatermark}
\makeatletter
\if@twocolumn
	\SetWatermarkText{Preprint}
	\SetWatermarkScale{0.20}
	\SetWatermarkAngle{270}
	%note: I defined this commands in the draftwatermark.sty file. For some reason they are in teh manual but not in the sty file...
	\SetWatermarkHorCenter{0.97\paperwidth}
	\SetWatermarkVerCenter{-.88\paperheight}
\else
	\SetWatermarkText{Teaching Concept}
	\SetWatermarkScale{0.25}
	\SetWatermarkAngle{270}
	%note: I defined this commands in the draftwatermark.sty file. For some reason they are in teh manual but not in the sty file...
	\SetWatermarkHorCenter{0.95\paperwidth}
	\SetWatermarkVerCenter{-.82\paperheight}
\fi

%uncomment the type of bioRxiv preprint below to added to the first page
\newresults{}
%\confirmatoryresults{}
%\contradictoryresults{}

%flushleft the title, authorblock and Abstract
%modified from http://tex.stackexchange.com/questions/85343/left-align-abstract-title-and-authors
\makeatletter
\renewcommand{\maketitle}{\bgroup\setlength{\parindent}{0pt}
\begin{flushleft}
  \thispagestyle{plain}
  \textbf{\@title}

  \@author
\end{flushleft}\egroup
}
\makeatother

%redefine the abstract
\renewenvironment{abstract}
 {\small
  \begin{flushleft}
  \textbf{\abstractname}\vspace{-0.40em}\vspace{0pt}
  \end{flushleft}
  \list{}{
    \setlength{\leftmargin}{0cm}%
    \setlength{\rightmargin}{\leftmargin}%
  }%
  \item\relax}
 {\endlist}

\hyphenation{}

\renewcommand*{\thefootnote}{\fnsymbol{footnote}}

%if to do notes need to be added: need to configure this!
\usepackage[colorinlistoftodos]{todonotes}

%feel free to change the reference style to suit your needs
\usepackage[firstinits=true, backref=false, maxcitenames=99, sorting=none, hyperref=auto, style=numeric-comp, defernumbers=true, backend=bibtex]{biblatex}[2010/11-19]

%change the name of this file to point to your bib file.
\bibliography{./Bibliography/Literature}

%send footnotes to the endlis
\usepackage{endnotes}

%global no indent
\setlength{\parindent}{0pt}

\begin{document}

%always keep the \newline command at the end of the title to add space between the title and the authors
\title{\Large Teaching Concept\newline}

\author[1]{Sergio Vargas R.\footnote[2]{\href{sergio.vargas@lmu.de}{sergio.vargas@lmu.de}}}
%\author[1,3,4]{Author with multiple affiliations}

\affil[1]{Department of Earth- \& Environmental Sciences, Palaeontology and Geobiology, Ludwig-Maximilians-Universtit\"at M\"unchen, Richard-Wagner Str. 10, D-80333 M\"unchen, Germany}
%\affil[2]{Forschungsinstitut und Naturmuseum Senckenberg, Senckenberganlage 25, D-60325 Frankfurt am Main, Germany}
%\affil[3]{Bavarian State Collections of Palaeontology and Geology, Richard-Wagner Str. 10, D-80333 M\"unchen, Germany}
%\affil[4]{GeoBio-CenterLMU, Richard-Wagner Str. 10, D-80333 M\"unchen, Germany}

\date{}

\maketitle
%\tableofcontents
%\begin{abstract}
%Blah, blah, blah.
%\\

%\textbf{Key words} Blah, Blah blah.
%\end{abstract}

After four years teaching different Bachelor and Master level courses at the LMU M\"unchen I see teaching as the best way to engage and motivate students to pursue a carrier in research. I also see it as an oportunity to include students into ongoing, sometimes incipient research projects and give them a learning experience based on real questions and real challenges.\\

I enjoy teaching, but in some courses (e.g.~statistics) keeping students motivated can be chanllenging. Because of this, I enrolled in the \underline{PROFI} in der \underline{L}ehre (PROFIL) continuing education program of the LMU. The PROFIL program offers training to university staff with a teaching duty. The seminars offered deal with different topics related to the conception of different types of courses, the deployment of different evaluation forms, student counseling, advising, mentoring and coaching, and other soft skills like the creation and use of flipcharts during courses.\\

To date I have participated in four seminars offered by PROFIL and obtained the \emph{``Zertifikat Hochschul\-lehre Bayern''}. In February and May, I will participate in two more seminars on student supervision and coaching. So far, the training helped me to improve how I conceive and plan my lectures and seminars, and raise my students' awareness about the importance of what they currently learn. For instance, for my bachelor level course on data analysis I try to demonstrate the value of open and transparent scientific practices, good, traceable data management and collaborative research. These are key values I am passionate about and that I try to use as much as possible for my own research. For this, \emph{in plenum}, we try to plan an experiment to replicate the results of a published paper the students read as homework. I use published papers with a succinct \emph{Materials and Methods} section to highlight the importance of clearly written methodologies, transparent data analysis pipelines, and open and freely accessible data and software have in scientific research. Once the case for traceability and openness has been made, the students learn to use \emph{git} as a version control system and Github to provide a transparent way to collaborate and share data, scripts and analytical pipelines. To include a hands-on part I use a class repository\footnote{See: \href{https://github.com/geobiolmu/WP49.2_Datenverarbeitung_WiSe1617}{https://github.com/geobiolmu/WP49.2\_Datenverarbeitung\_WiSe1617}} where the flipcharts and the data I use during the course are stored and the students are required to send their homeworks to. In this repository I also store the Jupyter\footnote{For more information on the Jupyter project see: \href{http://jupyter.org}{http://jupyter.org}} R Notebooks I use during the class to explain the concepts we cover and provide examples written in R that can be used as hands-on tutorials by the students. To evaluate this lecture I use a \emph{Portfolio} that consist of student scripts stored in the repository and that cover a number of basic operations in R, like plotting graphs, calculating descriptive statistics, and simple hypothesis testing. I have seen that alternative forms of evaluating the students' progress on a topic has a positive impact on the students' motivation and interest towards a topic.\\

My lecturing has benefited from the \emph{``M\"unchener Methodenkasten\footnote{\href{http://www.profil.uni-muenchen.de/angebote/muenchner-methodenkasten/index.html}{http://www.profil.uni-muenchen.de/angebote/muenchner-methodenkasten/index.html}}''}, a toolkit for university teachers which uses the AVIVA method to structure each lecture. I try to stick to this method as much as possible and plan each lecture based on it. I find it helps me structuring the lecture in a more adequate manner as it requires me to produce a detailed plan of the activities I want to do during a lecture. This way I have to reflect about the contents and goals of the lecture I am planning. I think having access to an activity toolkit that is easy to understand and deploy improves my time and content management and helps avoiding the classical one-way ``Vorlesung''. It also allow students to develop own ideas that can be reflected \emph{in plenum} with the other students, making a lecture more interactive.\\

I use some of the methods during my part of the evolution bachelor lecture. For instance I randomly assign students to one of the first four chapters of Darwin's ``On the Origin'', which they have to read. During the next class groups of students sharing the same chapter have to discuss the content of the chapter and produce a one sentence summary of the entire chapter. Once each chapter has its ``catchphrase'', these are discussed \emph{in plenum}. Then I try to link Darwin's text with the modern view on how variation originates by letting groups of students simulate the evolution of one nucleotide position using Markov models and compare the ``end product'' of the stochastic mutation process obtained by each group. We discuss the effect of the resulting mutation if it occurred in different regions of an eukaryotic gene (e.g.~Intron \emph{vs.}~Exon). Then I move from the molecular part to the population level to learn about how the variation generated through mutation spreads in a population and once these basic concepts are explained the course moves to more special topics, like coevolution, speciation, that overlap with my own research and offer me with an opportunity to communicate my research to the students. At the master level we cover similar topics but the students are more involved and typically have to give a lecture on a topic of their interest. This lecture is followed by a feedback session in which the presentation skills are analyzed by the group and general recommendations are given on how to prepare a talk; a discussion/clarification of the content of the presentation is also done. I like combining this student-driven talks with homeworks that consist in reading a scientific paper on the topic of the week and writing a press-release that summarizes to the layperson the relevance of the assigned article. This a good writing exercise that force students to understand and being able to explain the most important findings of the selected study; I give feedback using the track-changes function available in most text editors. I firmly believe that soft-skills like presenting, writing, creating posters, etc. have to be included in the \emph{curriculum} and practice constantly to improve the skills of the students, thus I try include exercises in this direction in all my curses.\\

I have taught different courses since I started my current position in 2013 and also had the opportunity to participate in the transition to new Bachelor and Master level \emph{curricula}. Thanks to this I was able to participate in the conception and design of new courses and get first-hand experience on the process of transforming \emph{curricula} and the associated challenges this involves (e.g.~Bologna-conformity, Module-conception). I think I benefited from the diversity of courses I had to teach and now have a good idea of the methods that may work well for different lecture formats (e.g.~Vorlesung \emph{vs.}~Seminar \emph{vs.} Laboratory practical) and number of students. I have also benefited from several opportunities I had to supervise Bachelor, Master and Research\footnote{We host students from the Erasmus Mundus Master Programme in Evolutionary Biology and the Munich Graduate School for Evolution, Ecology and Systematics who conduct their Research Training in the laboratory.} students and doctoral fellows. For these projects, I pursue to propose bachelor projects that complement master or doctoral studies. I think this increases the level of intra-group collaboration between students providing a supportive environment in which the new students get to learn from older ones and more senior members of the group have the opportunity to gain experience supervising students. It also allows a more tighter cohesion and easier supervision of the group as whole. I judge this strategy has worked well as revealed by the regular appearance of undergraduate coauthors in our last publications.\\

Finally, after revising the Bachelor and Master Program of the Dept. of Ecology \& Evolution, Uni-Salzburg, I think that a relatively fluid transition from my current position to a new position in Salzburg is possible, as many of the courses with a Zoological focus taught in Salzburg are either equivalent to courses I give or resemble courses I am familiar with from our programs. Thus, I feel confident I can contribute to the future development of the Biology program in Salzburg, complement the current faculty and learn from them.



%print note at the end
%\begingroup
%\parindent 0pt
%\def\enotesize{\footnotesize}
%\theendnotes
%\endgroup

%\listoftodos[To Do:]
\end{document}
