%Sergio Vargas R. 2015.

\documentclass[a4paper,11pt]{article}
%\documentclass[a4paper,11pt,twocolumn]{article}%twocolumn layout; not sure if this works correctly. Feel free to experiment.
\usepackage[pdftex]{color,graphicx}
\usepackage[T1]{fontenc}
\usepackage{pxfonts}

%color links to figs, etc.
%\usepackage[pdftex,colorlinks,urlcolor=blue,breaklinks]{hyperref}
\usepackage[pdftex,colorlinks]{hyperref}
\hypersetup{colorlinks,%
		    citecolor=black,%
		    filecolor=black,%
		    linkcolor=black,%
		    urlcolor=black,%
		    breaklinks=true,%
		    pdftex}
		    
%easily manipulate margins
\usepackage{geometry}
\makeatletter
\if@twocolumn%
	\geometry{twoside,
		paperwidth=210mm,
  		paperheight=297mm,
  		textheight=682pt,
  		textwidth=516pt,
  		centering,
  		headheight=50pt,
  		headsep=12pt,
  		footskip=18pt,
  		footnotesep=24pt plus 2pt minus 12pt,
 		columnsep=14pt}	
\else%if using twocolumns, I still need to modify the textwidth to accomodate the watermark
	\geometry{twoside,
		  paperwidth=210mm,
		  paperheight=297mm,
		  textheight=558pt,
		  textwidth=380pt,
		  centering,
		  headheight=50pt,
		  headsep=12pt,
		  footskip=18pt,
		  footnotesep=24pt plus 2pt minus 12pt}
\fi%

%change the font
\usepackage{charter}

%CC logos to add in the footer
\usepackage{ccicons}

%easily manipulate color
\usepackage{color}

%nicely print bioRxiv
\newcommand{\biorxiv}{\emph{bio{\color{red}R}$\chi$iv}}

%add biorxiv type of article using the package background
\usepackage{tikz}
\usepackage[firstpage=True]{background}

%use either of the following commands \newresults, \confirmatoryresults or \contradictoryresults to define the type of article in the first page
\newcommand{\newresults}{
\makeatletter
\if@twocolumn%
	\backgroundsetup{
		position={0.572\paperwidth,-0.025\paperheight} ,
		angle=270,
		color=black,
		opacity=0.60,
		scale=1.5,
		contents={\tikz\node[text=black,fill=gray!40,align=left, minimum width=2.3cm,minimum height=0.6cm,inner sep=0]{New Results};}}
\else
	\backgroundsetup{
		position={0.338\paperwidth,-0.06\paperheight} ,
		angle=270,
		color=black,
		opacity=0.60,
		scale=2.5,
		contents={\tikz\node[text=black,fill=gray!40,align=left, minimum width=6.0cm,minimum height=0.6cm,inner sep=0]{Salzburg, Evolutionary Zoology};}}
\fi%
}

\newcommand{\confirmatoryresults}{
\makeatletter
\if@twocolumn%
\backgroundsetup{
	position={0.572\paperwidth,-0.03\paperheight} ,
	angle=270,
	color=black,
	opacity=0.60,
	scale=1.5,
	contents={\tikz\node[text=black,fill=gray!40,align=left, minimum width=3.9cm,minimum height=0.6cm,inner sep=0]{Confirmatory Results};}}
\else
	\backgroundsetup{
	position={0.338\paperwidth,-0.07\paperheight} ,
	angle=270,
	color=black,
	opacity=0.60,
	scale=2.5,
	contents={\tikz\node[text=black,fill=gray!40,align=left, minimum width=3.9cm,minimum height=0.6cm,inner sep=0]{Confirmatory Results};}}
\fi%
}

\newcommand{\contradictoryresults}{
\makeatletter
\if@twocolumn%
	\backgroundsetup{
	position={0.572\paperwidth,-0.03\paperheight} ,
	angle=270,
	color=black,
	opacity=0.60,
	scale=1.5,
	contents={\tikz\node[text=black,fill=gray!40,align=left, minimum width=4.0cm,minimum height=0.6cm,inner sep=0]{Contradictory Results};}}
\else
	\backgroundsetup{
	position={0.338\paperwidth,-0.07\paperheight} ,
	angle=270,
	color=black,
	opacity=0.60,
	scale=2.5,
	contents={\tikz\node[text=black,fill=gray!40,align=left, minimum width=4.0cm,minimum height=0.6cm,inner sep=0]{Contradictory Results};}}
\fi%
}

%enhanced floats
\usepackage{float}

%rotate floats if necessary
\usepackage{rotating}

%nicer, clever tables, uncomment if necessary
\usepackage{supertabular}

%tables with notes, etc., uncomment if necessary
%\usepackage{threeparttable}

%improved captions, uncomment if necessary
\usepackage{caption}

%add an author block
\usepackage{authblk}
%redefine authorblock font size and affiliation font size
\renewcommand\Authfont{\large}
\renewcommand\Affilfont{\scriptsize}

%add nice headers
\usepackage{fancyhdr}
\pagestyle{fancy}
\renewcommand{\headrulewidth}{0pt}
\lhead{{\footnotesize Vargas, 2017}}
\chead{}
\rhead{{\footnotesize Salzburg Evolutionary Biology, Research Concept}}
\lfoot{\ccbynd}
\cfoot{\thepage}
\rfoot{}

%redefine plain header
\fancypagestyle{plain}{
\renewcommand{\headrulewidth}{0pt}
\lhead{}
\chead{}
\rhead{{\footnotesize}}
\lfoot{}
\cfoot{\ccbynd}
\rfoot{}
}

%add water mark to the bottom right of all pages to make clear this is a preprint
\usepackage{draftwatermark}
\makeatletter
\if@twocolumn
	\SetWatermarkText{Preprint}
	\SetWatermarkScale{0.20}
	\SetWatermarkAngle{270}
	%note: I defined this commands in the draftwatermark.sty file. For some reason they are in teh manual but not in the sty file...
	\SetWatermarkHorCenter{0.97\paperwidth}
	\SetWatermarkVerCenter{-.88\paperheight}
\else
	\SetWatermarkText{Research Concept}
	\SetWatermarkScale{0.25}
	\SetWatermarkAngle{270}
	%note: I defined this commands in the draftwatermark.sty file. For some reason they are in teh manual but not in the sty file...
	\SetWatermarkHorCenter{0.95\paperwidth}
	\SetWatermarkVerCenter{-.82\paperheight}
\fi

%uncomment the type of bioRxiv preprint below to added to the first page
\newresults{}
%\confirmatoryresults{}
%\contradictoryresults{}

%flushleft the title, authorblock and Abstract
%modified from http://tex.stackexchange.com/questions/85343/left-align-abstract-title-and-authors
\makeatletter
\renewcommand{\maketitle}{\bgroup\setlength{\parindent}{0pt}
\begin{flushleft}
  \thispagestyle{plain}
  \textbf{\@title}

  \@author
\end{flushleft}\egroup
}
\makeatother

%redefine the abstract
\renewenvironment{abstract}
 {\small
  \begin{flushleft}
  \textbf{\abstractname}\vspace{-0.40em}\vspace{0pt}
  \end{flushleft}
  \list{}{
    \setlength{\leftmargin}{0cm}%
    \setlength{\rightmargin}{\leftmargin}%
  }%
  \item\relax}
 {\endlist}

\hyphenation{}

\renewcommand*{\thefootnote}{\fnsymbol{footnote}}

%if to do notes need to be added: need to configure this!
\usepackage[colorinlistoftodos]{todonotes}

%feel free to change the reference style to suit your needs
\usepackage[firstinits=true, backref=false, maxcitenames=99, sorting=none, hyperref=auto, style=numeric-comp, defernumbers=true, backend=bibtex]{biblatex}[2010/11-19]

%change the name of this file to point to your bib file.
\bibliography{./Bibliography/Literature}

%global no indent
\setlength{\parindent}{0pt}

\begin{document}

%always keep the \newline command at the end of the title to add space between the title and the authors
\title{\Large Research Concept\newline}

\author[1]{Sergio Vargas \footnote[2]{\href{sergio.vargas@lmu.de}{sergio.vargas@lmu.de}}}
%\author[1,3,4]{Author with multiple affiliations}

\affil[1]{Department of Earth- \& Environmental Sciences, Palaeontology and Geobiology, Ludwig-Maximilians-Universtit\"at M\"unchen, Richard-Wagner Str. 10, D-80333 M\"unchen, Germany}
%\affil[2]{Forschungsinstitut und Naturmuseum Senckenberg, Senckenberganlage 25, D-60325 Frankfurt am Main, Germany}
%\affil[3]{Bavarian State Collections of Palaeontology and Geology, Richard-Wagner Str. 10, D-80333 M\"unchen, Germany}
%\affil[4]{GeoBio-CenterLMU, Richard-Wagner Str. 10, D-80333 M\"unchen, Germany}

\date{}

\maketitle
%\tableofcontents
%\begin{abstract}
%Blah, blah, blah.
%\\

%\textbf{Key words} Blah, Blah blah.
%\end{abstract}

My research can be divided in two main lines I draw when I started with my habilitation in 2014. Both try to leverage new sequencing technologies to answer evolutionary question in octocorals and sponges.

\section*{Speciation and Adaptation genomics of aquatic invertebrates}

Perhaps the continuation of my early research on the systematics of octocorals is represented by this research line. I have worked for about 10 years with the octocoral fauna of the eastern Pacific and during the past three years extended my research to the Mediterranean, which offer many interesting possibilities in terms of the study of the adaptation genomics of octocorals.\\

Since my master thesis I have been interested in species delimitation in \emph{Pacifigorgia}, a species-rich gorgonian genus from the eastern Pacific. \emph{Pacifigorgia} is interesting in several ways, but the distribution of its centers of  endemism is, perhaps, the most intriguing aspect of the biology of this genus. Several endemic species have been described from the Gulf of Chiriqu\'i, while the Gulf of Panam\'a and the coast of Costa Rica is less species-rich and mostly inhabited by wide-spread species. In 2013 I secured funding from the LMU Excellent Junior Funds program to study the speciation genomics in \emph{Pacifigorgia}. As part of a collaborative effort, colleages in Costa Rica and Panama collected several species of \emph{Pacifigorgia} in the Gulf of Chiriqu\'i and we use reduce representation libraries to rapidly gather single-nucleotide-polymorphisms (SNPs) in several \emph{Pacifigorgia} species. The data generated was used by our former doctoral student to delimite \emph{Pacifigorgia} using a number of different methods. Interestingly we were not able to separate most species using our SNP dataset and we are currently working towards a manuscript reporting these findings that may indicate that the speciation process in \emph{Pacifigorgia} is still ongoing.\\

Using the same molecular methods we generated a SNP dataset for the Mediterranean octocoral \emph{Leptogorgia sarmentosa}. This dataset will be used to scan the transcriptome of \emph{L.~sarmentosa} to discover loci under selection and evaluate a possible link between the genes under selection and the environmental regime to which the individual corals are exposed. In addition to our SNP dataset shallow genome sequencing of this species was done to estimate the size of its genome and evaluate different strategies to sequence a draft genome of this species. \emph{Leptogorgia sarmentosa} is the only representative of the genus in the Mediterranean. Our own research has shown that this species resulted from a recent speciation event that separated it from Atlantic \emph{Leptogorgia} species. The extreme temperature regimes that exist in the Mediterranean, which contrast with the conditions prevailing in the Atlantic, make \emph{L. sarmentosa} an attractive model to study how the genome of this species drifted to adapt to the variable enviromental conditions in the Mediterranean compared to the Atlantics (e.g.~by sequencing the genome of \emph{L. sarmentosa} sister species).\\


\section*{Evolutionary consequences of symbiont-host interactions in sponges}




%\section{Expertise, background and completed projects}


%\section{Ongoing reseach projects}


%\section{Closed and ongoing student-driven research projects}


%\section{Research prospects}


%\todo[inline]{this is how you can add todo notes to your manuscript!...}

%if you want to add the references manually uncomment the following lines
%\section*{References}
%{\noindent
%Your reference here... if you want to write them.
%}

%you also need to comment the line \bibliography{./Bibliography/Literature} in the preamble...

%if you use bibtex use it to print the bibliography
%\printbibliography

%Paste tables and figures here if you want them to appear at the end of the preprint. Otherwise place them wherever you want!

\newpage

%\listoftodos[To Do:]

\end{document}
