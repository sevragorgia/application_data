%Sergio Vargas R. 2015.

\documentclass[a4paper,11pt]{article}
%\documentclass[a4paper,11pt,twocolumn]{article}%twocolumn layout; not sure if this works correctly. Feel free to experiment.
\usepackage[pdftex]{color,graphicx}
\usepackage[T1]{fontenc}
\usepackage{pxfonts}

%color links to figs, etc.
%\usepackage[pdftex,colorlinks,urlcolor=blue,breaklinks]{hyperref}
\usepackage[pdftex,colorlinks]{hyperref}
\hypersetup{colorlinks,%
		    citecolor=black,%
		    filecolor=black,%
		    linkcolor=black,%
		    urlcolor=black,%
		    breaklinks=true,%
		    pdftex}
		    
%easily manipulate margins
\usepackage{geometry}
\makeatletter
\if@twocolumn%
	\geometry{twoside,
		paperwidth=210mm,
  		paperheight=297mm,
  		textheight=682pt,
  		textwidth=516pt,
  		centering,
  		headheight=50pt,
  		headsep=12pt,
  		footskip=18pt,
  		footnotesep=24pt plus 2pt minus 12pt,
 		columnsep=14pt}	
\else%if using twocolumns, I still need to modify the textwidth to accomodate the watermark
	\geometry{twoside,
		  paperwidth=210mm,
		  paperheight=297mm,
		  textheight=558pt,
		  textwidth=380pt,
		  centering,
		  headheight=50pt,
		  headsep=12pt,
		  footskip=18pt,
		  footnotesep=24pt plus 2pt minus 12pt}
\fi%

%change the font
\usepackage{charter}

%CC logos to add in the footer
\usepackage{ccicons}

%easily manipulate color
\usepackage{color}

%nicely print bioRxiv
\newcommand{\biorxiv}{\emph{bio{\color{red}R}$\chi$iv}}

%add biorxiv type of article using the package background
\usepackage{tikz}
\usepackage[firstpage=True]{background}

%use either of the following commands \newresults, \confirmatoryresults or \contradictoryresults to define the type of article in the first page
\newcommand{\newresults}{
\makeatletter
\if@twocolumn%
	\backgroundsetup{
		position={0.572\paperwidth,-0.025\paperheight} ,
		angle=270,
		color=black,
		opacity=0.60,
		scale=1.5,
		contents={\tikz\node[text=black,fill=gray!40,align=left, minimum width=2.3cm,minimum height=0.6cm,inner sep=0]{New Results};}}
\else
	\backgroundsetup{
		position={0.338\paperwidth,-0.06\paperheight} ,
		angle=270,
		color=black,
		opacity=0.60,
		scale=2.5,
		contents={\tikz\node[text=black,fill=gray!40,align=left, minimum width=6.0cm,minimum height=0.6cm,inner sep=0]{Salzburg, Evolutionary Zoology};}}
\fi%
}

\newcommand{\confirmatoryresults}{
\makeatletter
\if@twocolumn%
\backgroundsetup{
	position={0.572\paperwidth,-0.03\paperheight} ,
	angle=270,
	color=black,
	opacity=0.60,
	scale=1.5,
	contents={\tikz\node[text=black,fill=gray!40,align=left, minimum width=3.9cm,minimum height=0.6cm,inner sep=0]{Confirmatory Results};}}
\else
	\backgroundsetup{
	position={0.338\paperwidth,-0.07\paperheight} ,
	angle=270,
	color=black,
	opacity=0.60,
	scale=2.5,
	contents={\tikz\node[text=black,fill=gray!40,align=left, minimum width=3.9cm,minimum height=0.6cm,inner sep=0]{Confirmatory Results};}}
\fi%
}

\newcommand{\contradictoryresults}{
\makeatletter
\if@twocolumn%
	\backgroundsetup{
	position={0.572\paperwidth,-0.03\paperheight} ,
	angle=270,
	color=black,
	opacity=0.60,
	scale=1.5,
	contents={\tikz\node[text=black,fill=gray!40,align=left, minimum width=4.0cm,minimum height=0.6cm,inner sep=0]{Contradictory Results};}}
\else
	\backgroundsetup{
	position={0.338\paperwidth,-0.07\paperheight} ,
	angle=270,
	color=black,
	opacity=0.60,
	scale=2.5,
	contents={\tikz\node[text=black,fill=gray!40,align=left, minimum width=4.0cm,minimum height=0.6cm,inner sep=0]{Contradictory Results};}}
\fi%
}

%enhanced floats
\usepackage{float}

%rotate floats if necessary
\usepackage{rotating}

%nicer, clever tables, uncomment if necessary
\usepackage{supertabular}

%tables with notes, etc., uncomment if necessary
%\usepackage{threeparttable}

%improved captions, uncomment if necessary
\usepackage{caption}

%add an author block
\usepackage{authblk}
%redefine authorblock font size and affiliation font size
\renewcommand\Authfont{\large}
\renewcommand\Affilfont{\scriptsize}

%add nice headers
\usepackage{fancyhdr}
\pagestyle{fancy}
\renewcommand{\headrulewidth}{0pt}
\lhead{{\footnotesize Vargas, 2017}}
\chead{}
\rhead{{\footnotesize Salzburg Evolutionary Biology, Research Concept}}
\lfoot{\ccbynd}
\cfoot{\thepage}
\rfoot{}

%redefine plain header
\fancypagestyle{plain}{
\renewcommand{\headrulewidth}{0pt}
\lhead{}
\chead{}
\rhead{{\footnotesize}}
\lfoot{}
\cfoot{\ccbynd}
\rfoot{}
}

%add water mark to the bottom right of all pages to make clear this is a preprint
\usepackage{draftwatermark}
\makeatletter
\if@twocolumn
	\SetWatermarkText{Preprint}
	\SetWatermarkScale{0.20}
	\SetWatermarkAngle{270}
	%note: I defined this commands in the draftwatermark.sty file. For some reason they are in teh manual but not in the sty file...
	\SetWatermarkHorCenter{0.97\paperwidth}
	\SetWatermarkVerCenter{-.88\paperheight}
\else
	\SetWatermarkText{Research Concept}
	\SetWatermarkScale{0.25}
	\SetWatermarkAngle{270}
	%note: I defined this commands in the draftwatermark.sty file. For some reason they are in teh manual but not in the sty file...
	\SetWatermarkHorCenter{0.95\paperwidth}
	\SetWatermarkVerCenter{-.82\paperheight}
\fi

%uncomment the type of bioRxiv preprint below to added to the first page
\newresults{}
%\confirmatoryresults{}
%\contradictoryresults{}

%flushleft the title, authorblock and Abstract
%modified from http://tex.stackexchange.com/questions/85343/left-align-abstract-title-and-authors
\makeatletter
\renewcommand{\maketitle}{\bgroup\setlength{\parindent}{0pt}
\begin{flushleft}
  \thispagestyle{plain}
  \textbf{\@title}

  \@author
\end{flushleft}\egroup
}
\makeatother

%redefine the abstract
\renewenvironment{abstract}
 {\small
  \begin{flushleft}
  \textbf{\abstractname}\vspace{-0.40em}\vspace{0pt}
  \end{flushleft}
  \list{}{
    \setlength{\leftmargin}{0cm}%
    \setlength{\rightmargin}{\leftmargin}%
  }%
  \item\relax}
 {\endlist}

\hyphenation{}

\renewcommand*{\thefootnote}{\fnsymbol{footnote}}

%if to do notes need to be added: need to configure this!
\usepackage[colorinlistoftodos]{todonotes}

%feel free to change the reference style to suit your needs
\usepackage[firstinits=true, backref=false, maxcitenames=99, sorting=none, hyperref=auto, style=numeric-comp, defernumbers=true, backend=bibtex]{biblatex}[2010/11-19]

%change the name of this file to point to your bib file.
\bibliography{./Bibliography/Literature}

%global no indent
\setlength{\parindent}{0pt}

\begin{document}

%always keep the \newline command at the end of the title to add space between the title and the authors
\title{\Large Research Concept\newline}

\author[1]{Sergio Vargas \footnote[2]{\href{sergio.vargas@lmu.de}{sergio.vargas@lmu.de}}}
%\author[1,3,4]{Author with multiple affiliations}

\affil[1]{Department of Earth- \& Environmental Sciences, Palaeontology and Geobiology, Ludwig-Maximilians-Universtit\"at M\"unchen, Richard-Wagner Str. 10, D-80333 M\"unchen, Germany}
%\affil[2]{Forschungsinstitut und Naturmuseum Senckenberg, Senckenberganlage 25, D-60325 Frankfurt am Main, Germany}
%\affil[3]{Bavarian State Collections of Palaeontology and Geology, Richard-Wagner Str. 10, D-80333 M\"unchen, Germany}
%\affil[4]{GeoBio-CenterLMU, Richard-Wagner Str. 10, D-80333 M\"unchen, Germany}

\date{}

\maketitle
%\tableofcontents
%\begin{abstract}
%Blah, blah, blah.
%\\

%\textbf{Key words} Blah, Blah blah.
%\end{abstract}

My research can be divided in two main lines I drew when I started my Habilitation in 2014. Both try to leverage new (short- and long-read) sequencing technologies to answer evolutionary question in sponges and octocorals. I am interested in understanding how environmental change related stress factors affect the sponge holobiont and whether variations in the symbiotic community of sponges may lead to an increase in the fitness of the entire system. I am also interested in the role played by gene flow as a force that can counteract natural selection at a local level in octocorals and sponges.\\

Below I outline the main research topics I would like to further develop in Salzburg.

\section*{Holobiont adaptomics and response to climate change}
\subsection*{Freshwater sponges}

\emph{Spongilla lacustris} is a freshwater sponge commonly found across Europe. Freshwater sponges produce special overwintering structures called ``gemmulae'' that can be collected and kept in the laboratory in a dormant state for long periods of time. Sponges can be produced by letting the gemmulae hatch, and the hatched sponges are easy to keep in the laboratory for experimentation.\\

Only scarce information exist about the bacterial symbiotic community associated with \emph{S.~lacustris}\footnote{Gernert et al.~2005.~Microb.~Ecol.~50:206-212} and no information exist to date on the geographical variation of the \emph{S.~lacustris} microbiome. In addition, within a single population \emph{S.~lacustris} can be found harboring unicellular algae of the genus \emph{Chlorella} or, on the contrary, the algae may not be present. The effect of presence/absence of \emph{Chlorella} on the symbiotic bacterial community of \emph{S.~lacustris} also remains to be determined. \\

I would like to conduct future research on this system to assess:
\begin{enumerate}
\item what are the connectivity patterns between populations of \emph{S.~lacustris},
\item how is natural selection at local scales, due to exposure to contrasting local environments (e.g.~alpine lakes \emph{vs.}~Mediterranean freshwater systems), shaping genome evolution in different populations of \emph{S.~lacustris},
\item how does the microbiome of \emph{S.~lacustris} change along latitudinal and environmental gradients, and
\item how do different \emph{S.~lacustris} holobionts respond and adapt to climate-change induced stress factors.
\end{enumerate}

To answer these questions, I would like to establish a collection of \emph{S.~lacustris} from multiple localities in Europe. Alpine aquatic environments would provide a unique opportunity due to the sometimes marked differences observed between closely located water bodies. We have had a good experience collecting \emph{S.~lacustris} in Lake Constance and know from populations in other lakes. Contact with hydroelectric power plant operators that may regard these sponges biofouling agents, may also provide valuable research opportunities. \\

As part of my research on sponge genomics and transcriptomics, I have produced genomic data that could be used in the future to assemble a draft genome of this species. In addition, I currently have access to material from the Bodensee that will be used during the next Spring for the preliminary characterization of the microbiome of \emph{S.~lacustris}. These data will serve as foundation for future research on this system.

\subsection*{Cyanosponges}

Cyanosponges are sponges that have a symbiotic association with cyanobacteria. These sponges occur in shallow marine environments world-wide. I have been working with a common cyanosponge that can be easily cultured and manipulated in marine aquaria. Curiously, the taxonomic affiliation of this aquarium sponge is not well established yet.\\

With this system I have tried to characterize the microbial community inhabiting the sponge and currently pursue to close some of the symbiont genomes. I have assessed the response of the symbiotic community to bleaching (i.e.~loss of the symbiotic cyanobacteria). I have also assessed how bleaching affects the sponge transcriptome. I am currently analyzing the data I have produced and writing the manuscripts that communicate the results of this project, which represents a large part of my habilitation research.

Future research will use this system as a tractable model sponge holobiont for the study of the molecular basis of symbiont-host interaction and of the effects of climate-induced stress factors on it. I have already identified a number of candidate genes that constitute a basis for future experiments involving, for instance, \emph{in situ} hybridization to locate their expression and test whether cells actively expressing these genes are interacting with the symbionts or not.

\section*{Adaptation genomics in \emph{Leptogorgia sarmentosa}}

The past two years we have used reduce representation libraries to rapidly gather genomic information of different species of octocorals and sponges. We worked first with eastern Pacific and later with Mediterranean octocorals, trying to move my research area to Europe as it is logistically easier to reach the Mediterranean than the eastern Pacific. In 2013 I secured funding from the LMU Excellent Junior Funds program to study the speciation genomics in \emph{Pacifigorgia}. Thanks to this funding our former doctoral student Dr. Angelo Poliseno generated SNP datasets for several species of \emph{Pacifigorgia} and for the Mediterranean octocoral \emph{Leptogorgia sarmentosa}.\\

\emph{Leptogorgia sarmentosa} is the only representative of the genus in the Mediterranean. Our research shows that this species resulted from a recent speciation event that separated it from Atlantic \emph{Leptogorgia} species. The variable temperature regimes to which \emph{L.~sarmentosa} is exposed make this species an attractive model to study how its genome adapts to different enviromental conditions across the Mediterranean. With the data currently available coding regions of the genome under selection in \emph{L.~sarmentosa} can be identified. The easy access to \emph{L.~sarmentosa} populations under different water temperature regimes (e.g.~Cadiz vs. the Adriatic) makes this species an ideal target for field studies designed to determine how the expression of genes under selection change between and within populations and whether these local adaptations increase the survival potential of the octocorals. In addition, the role of gene flow, as a force acting against natural selection, can be studied given the restricted geographic distribution of \emph{L.~sarmentosa} and relative discrete populations. I produced low coverage genome sequence data with which a genome size estimation can done and a strategy to sequence a (draft) genome of this species can be developed.\\




%\section{Expertise, background and completed projects}


%\section{Ongoing reseach projects}


%\section{Closed and ongoing student-driven research projects}


%\section{Research prospects}


%\todo[inline]{this is how you can add todo notes to your manuscript!...}

%if you want to add the references manually uncomment the following lines
%\section*{References}
%{\noindent
%Your reference here... if you want to write them.
%}

%you also need to comment the line \bibliography{./Bibliography/Literature} in the preamble...

%if you use bibtex use it to print the bibliography
%\printbibliography

%Paste tables and figures here if you want them to appear at the end of the preprint. Otherwise place them wherever you want!

\newpage

%\listoftodos[To Do:]

\end{document}
